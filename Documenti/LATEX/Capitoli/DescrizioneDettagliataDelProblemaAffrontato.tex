\counterwithin{figure}{section}

\chapter{Descrizione dettagliata del problema affrontato}

\label{Capitolo 2}

Il campionato di F1 è il campionato di spicco nel mondo del motorsport: i 20 piloti più abili al mondo gareggiano sulle vetture da corsa più performanti per vincere il campionato mondiale. Questo palcoscenico offre, più che una sfida tra le abilità dei piloti in pista, una sfida ingegneristica e strategica tra le scuderie. In F1, la maggior parte delle volte, non vince il miglior pilota in pista ma quello che guida la vettura più veloce e più affidabile.
Per questo motivo il lavoro dietro le quinte di una stagione di Formula 1 è durissimo, volto continuamente ad un’estrema innovazione tecnica di concetti aerodinamici, motoristici e di efficienza. 
Ma dal 2021, al fine di rendere la competizione più equa, è stata introdotta una regola che impone un tetto massimo di spese per gli investimenti tecnici per migliorare la vettura. Fino ad allora spesso trionfava chi investiva più fondi nella ricerca e nello sviluppo della vettura, ma dall’emissione del nuovo regolamento la sfida tecnica si è fatta più avvincente.
Ciò ha reso la sfida ingegneristica e gestionale molto intensa, con i vari team che si vedono costretti a scegliere in quali aree di ricerca investire maggiormente il denaro. 
L’obiettivo dell’applicazione è quello di individuare il più vantaggioso asset di investimenti per massimizzare il rendimento della propria squadra e quindi aumentare le percentuali di successo nel campionato. 
Questo sarà possibile svolgendo un gran numero di simulazioni e verificando quanto ogni diversa combinazione di asset possa apportare cambiamenti sul rendimento finale.
Non vi è però un asset univoco per massimizzare le qualità di una vettura di F1: nella competizione sono presenti dieci tipologie di vetture con, di conseguenza, diverse qualità tecniche tra di loro.
Ogni vettura ha un’area tecnica in cui è più carente rispetto al livello medio delle scuderie ma non è detto che lo sviluppo di quell’area sia il più fruttuoso in termini di prestazioni.
L’applicazione permette quindi di svolgere un gran numero di simulazioni, scegliendo accuratamente dove e come investire i 140 milioni di dollari per poi visualizzare i risultati complessivi di fine stagione. Questi saranno visualizzati tramite le righe di una colonna che renderà semplice e immediato il confronto tra le simulazioni.
Non è da sottovalutare la scelta dei piloti: quando il livello delle vetture è vicino tra loro, è la scuderia con la miglior coppia di piloti ad avere la meglio. 
L’applicazione permette anche di scegliere la coppia di piloti per la scuderia selezionata: il giusto pilota permetterà alla scuderia di sovraperformare e raggiungere grandi risultati.
L’applicazione calibrerà il peso dell’investimento anche in base all’arretratezza tecnica del team scelto: è più facile e meno rischioso innovarsi ispirandosi a concetti già presenti in team rivali invece che ricercare nuove soluzioni.
